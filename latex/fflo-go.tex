

\documentclass{article}
\usepackage[utf8]{inputenc}
\usepackage{setspace}
\usepackage{ mathrsfs }
\usepackage{amssymb} %maths
\usepackage{amsmath} %maths
\usepackage[margin=0.2in]{geometry}
\usepackage{graphicx}
\usepackage{ulem}
\setlength{\parindent}{0pt}
\setlength{\parskip}{10pt}
\usepackage{hyperref}
\usepackage[autostyle]{csquotes}

\usepackage{cancel}
\renewcommand{\i}{\textit}
\renewcommand{\b}{\textbf}
\newcommand{\q}{\enquote}
%\vskip1.0in



\begin{document}

\begin{huge}

{\setstretch{0.0}{
Fflo [ Go Version ]

This program displays the evolution of a continuous cellular automaton which is itself randomly evolving. What appears as a rectangular grid is computationally a torus, featuring both vertical and horizontal wraparound. This version has important extra features. Users can save and load filter settings called \q{cartridges.}

Filters and resolution are controlled using left and right shift keys and the number keys. For instance, holding \b{right shift} and pressing and releasing \b{8} will switch the system to using 8 filters. But holding \b{left shift} and pressing and releasing \b{8} will randomize only the eighth filter (if there is one.) Using the \b{left control} key with the number keys allows for a change of pixel resolution. 


Press and release \b{S} to reseed the environment, and use \b{F} or \b{P} to randomize the filters. Pressing and releasing \b{P} also randomizes the \q{power} (roughly, contrast) of the filters.


}}
\end{huge}
\end{document}
